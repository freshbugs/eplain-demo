\input eplain
\beginpackages
  \usepackage{url}
  \usepackage{color}
  \usepackage{graphicx}
\endpackages

This is a demo of some things you can do with eplain.tex.

To do these things in pdftex, see the example at
\verbatim
https://www.ctan.org/tex-archive/systems/doc/pdftex/samplepdftex
|endverbatim

You can also do them to be typeset by tex then dvipdfm.

You can
typeset this with pdftex or tex and then dvipdfm.

\beginsection Graphics

You can use pdfximage to include graphics in pdftex.  You can use
``specials'' to tex and then dvipdfm.  The dvipdfm driver is then
the one to insert the graphics.  For example,
if you have a two-by-one.pdf:

\verbatim
\vbox to 1in{\vss \special{pdf:image width 2in (sbigelow.pdf)}}
|endverbatim

For pdftex, use
\verbatim
\pdfximage width 2in {sbigelow.pdf} \pdfrefximage\pdflastximage
|endverbatim

Here I used eplain and the graphicx package.  It seems like tex (not
pdftex) needs me to tell it the natwidth and natheight.  Despite its
filename, this png seems to have a height of 311.  Maybe related: the
image comes out smaller with tex and dvipdfm than pdftex.

\ifpdf
\includegraphics[width=2in]{ctan_lion_350x350}
\else
\includegraphics[width=2in,natwidth=350,natheight=311]{ctan_lion_350x350.png}
\fi

\beginsection Tikz

For tex then dvipdfm (not pdftex), the tikz manual says to define
pgfsysdriver before inputting tikz.

\ifpdf \else \def\pgfsysdriver{pgfsys-dvipdfm.def} \fi
\input tikz

Here is a curved line.
\tikzpicture \draw (0,0) .. controls (0,1) .. (1,1); \endtikzpicture

\beginsection Hyperlinks

You can get hyperlinks in pdftex using
the pdfstartlink, but this is a low level command
and I don't know how to use it.

If you're using tex then dvipdfm,
you could do links with "specials":

\verbatim
\special{color push rgb 0.5 0.5 1}%
\special{html:<a href="https://arxiv.org">}%
\underbar{arXiv}%
\special{html:</a>}%
\special{color pop}
|endverbatim

Or, input eplain, and enable hyperlinks.  Seems like I need to define
their color otherwise I get an error.

\enablehyperlinks
\definecolor{urlcolor}{rgb}{.2,.4,.6}
\hlopts{colormodel=,color=urlcolor}

Here is a useful reference.
\url{https://www.tug.org/TUGboat/tb26-3/tb84katsi.pdf}

\beginsection References

Here I will use ``thm'' as a type of reference, and define thmword
to be ``Theorem''.

\def\thmword{Theorem}

Let's refer forward to a theorem labeled \ref{key}.

I could use a counter to call it Theorem 4.1 or whatever, but let's
just call it A.

\definexref{key}{A}{thm}
\proclaim Theorem A.
Every even number is a sum of two odd numbers.

\beginsection Citations

Here are the contents of the file mybib.bib.

\verbatim
@BOOK{Nobody,
AUTHOR="Nobody",
TITLE="The Book without Title",
PUBLISHER="Dummy Publisher",
YEAR="2100",
}
|endverbatim

Now I can do a citation for \cite{Nobody}.

\beginsection Bibliography

\bibliography{mybib}
\bibliographystyle{plain}

\bye
